% REQUIRED (remove if already imported smw else)
\usepackage{tikz}
\usetikzlibrary{decorations.pathreplacing}
\usepackage{environ} % sizing pictures properly
\usepackage[outline]{contour} % white outlines around text -- NOTE: might break if not using pdfLaTeX
\usetikzlibrary{patterns}
\usetikzlibrary{calc}
\tikzset{lab/.style={draw,rectangle,inner sep=0}}

% SETTINGS: colours and thickness
\definecolor{base}{rgb}{1,0,0} % red
%\definecolor{base}{rgb}{0.7, 0.7, 0.7} % grey

\definecolor{darkbase}{rgb}{0.8,0,0} % dark red
%\definecolour{darkbase}{rgb}{0.9,0.9,0.9} % dark grey

% outline thickness
\contourlength{0.1em}


%%% INSTRUCTIONS

% use everything in a begin{compline} block

% base commands: \drawbase          (empty rectangle)
%                \drawbaseunknown   (rectangle filled with unknown pattern)
%                \drawbasefade      (faded color areas) (nice but not practical)

% gap{start}{end}: creates a filled in white rectangle for a gap
% dense{start}{end}: creates a filled in [base colour] rectangle for density
% unknown{start}{end}: creates a dot pattern for density

% ADD TICKS AFTER FILL

% problem{position(0-18)}{depth}{label}: adds a thicker dark [base colour] tick at position

% emptyproblem: as above, no label or layer

% reference{start}{end}{label}: adds overbrace with caption over top; use for citing where the works are from

% probreference{pos}{extra height}{problem}{reference}: adds line with two line caption on top, use for citing examples of tight classes. extra height might be required for spacing

% multiproblem{start}{end}{depth}{xshift}{label line 1}{label line 2}: adds [base colour] underbrace for infinitely dense classes

% drawtick, tick, emptytick, addbaseticks <- DEPRECATED TO REMOVE

%%% CODE

% scaled pictures
\makeatletter
\newsavebox{\measure@tikzpicture}
\NewEnviron{scaletikzpicturetowidth}[1]{%
  \def\tikz@width{#1}%
  \def\tikzscale{1}\begin{lrbox}{\measure@tikzpicture}%
  \BODY
  \end{lrbox}%
  \pgfmathparse{#1/\wd\measure@tikzpicture}%
  \edef\tikzscale{\pgfmathresult}%
  \BODY
}
\makeatother

%\newenvironment{compline}[1][0.99]{%
\newenvironment{compline}{%
    \begin{center}%
    \begin{scaletikzpicturetowidth}{0.99\textwidth}%
    \begin{tikzpicture}[scale=\tikzscale]%
    \drawbaseunknown%
    }{%
    \end{tikzpicture}%
    \end{scaletikzpicturetowidth}%
    \end{center}%
    }
    
\newcommand{\emptyproblem}[1]{\draw[ultra thick, darkbase] (#1,0.5) -- (#1,0);}

\newcommand{\problem}[3]{\draw[ultra thick, darkbase] (#1,0.5) -- (#1,0) -- ++($(0,-2ex)-(0,#2)$);%
    \node (a) at ($(#1,-#2)-(0,3.5ex)$) {\contour{white}{#3}};%
    }
% ^ tried to make these both into a single 4 argument command, and it broke for some reason. So they're different for now

\newcommand{\multiproblem}[6]{%
    \draw[decorate,decoration={brace,raise=0.5ex,amplitude=2ex}, thick, red] (#2,0) --  (#1,0);%
    \coordinate (CENTER) at ($(#1,-0.5)!0.5!(#2,-0.5)+(0,0.5ex)$);%
    \draw[thick,red] (CENTER) -- ++(0,-0.1) -- ++(0,-#3);%
    \coordinate (LABEL) at ($(#1,0)!0.5!(#2,0)+(0,-0.8)+(#4,-#3)$);%
    \node (a) at (LABEL) {\contour{white}{#5}};%
    \node (b) at ($(LABEL)-(0,3ex)$) {\contour{white}{#6}};%
    }

\newcommand{\reference}[3]{%
    \draw[decorate,decoration={brace,raise=0.5ex,amplitude=2ex}, thick] (#1,0.5) --  (#2,0.5);%
    \coordinate (CENTER) at ($(#1,1)!0.5!(#2,1)-(0,0.15ex)$);%
    \coordinate (LABEL) at ($(#1,1)!0.5!(#2,1)+(0,0.2)$);%
    \draw[thick] (CENTER) -- (LABEL);%
    \node (a) at (LABEL) {\contour{white}{#3}};%
    }

\newcommand{\probreference}[4]{%
    \coordinate (LABEL) at ($(#1,1.5)+(0,#2)$);%
    \draw[thick] (#1,0.5) -- (LABEL);%
    \node (a) at ($(#1,#2)+(0,1.5)+(0,4ex)$) {\contour{white}{#3}};%
    \node (a) at ($(#1,#2)+(0,1.5)+(0,1ex)$) {\contour{white}{#4}};%
    }

\newcommand{\baseticks}[1]{%
    \tick{0}{#1}{$O(1)$}%
    \tick{6}{#1}{$\Theta(\log^*n)$}%
    \tick{12}{#1}{$\Theta(\log n)$}%
    \tick{18}{#1}{$\Theta(n)$}%
}

\newcommand{\dense}[2]{%
    \draw[fill=red, draw=none] ($(#1,0)+(0.075mm,0.075mm)$) rectangle ($(#2,0.5)-(0.075mm,0.075mm)$);%
}

\newcommand{\gap}[2]{%
    \draw[fill=white, draw=none] ($(#1,0)+(0.075mm,0.075mm)$) rectangle ($(#2,0.5)-(0.075mm,0.075mm)$);%
}

\newcommand{\unknown}[2]{%
    \pgfsetfillpattern{crosshatch dots}{base}%
    \filldraw[draw=none] (#1,0) rectangle (#2,0.5);%
}

\newcommand{\drawbase}{%
    \draw (0,0) rectangle (18,0.5);%
}

\newcommand{\drawbaseunknown}{%
    \pgfsetfillpattern{crosshatch dots}{base}%
    \filldraw (0,0) rectangle (18,0.5);%
}