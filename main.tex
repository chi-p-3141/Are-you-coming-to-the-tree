% !TeX TXS-program:bibliography = txs:///biber !TEX program = xelatex
\documentclass[12pt,a4paper]{article}
\input{Preamble/packages}
\input{Preamble/commands}
\addbibresource{ref.bib}

% "package" for complexity lines
% REQUIRED (remove if already imported smw else)
\usepackage{tikz}
\usetikzlibrary{decorations.pathreplacing}
\usepackage{environ} % sizing pictures properly
\usepackage[outline]{contour} % white outlines around text -- NOTE: might break if not using pdfLaTeX
\usetikzlibrary{patterns}
\usetikzlibrary{calc}
\tikzset{lab/.style={draw,rectangle,inner sep=0}}

% SETTINGS: colours and thickness
\definecolor{base}{rgb}{1,0,0} % red
%\definecolor{base}{rgb}{0.7, 0.7, 0.7} % grey

\definecolor{darkbase}{rgb}{0.8,0,0} % dark red
%\definecolour{darkbase}{rgb}{0.9,0.9,0.9} % dark grey

% outline thickness
\contourlength{0.1em}


%%% INSTRUCTIONS

% use everything in a begin{compline} block

% base commands: \drawbase          (empty rectangle)
%                \drawbaseunknown   (rectangle filled with unknown pattern)
%                \drawbasefade      (faded color areas) (nice but not practical)

% gap{start}{end}: creates a filled in white rectangle for a gap
% dense{start}{end}: creates a filled in [base colour] rectangle for density
% unknown{start}{end}: creates a dot pattern for density

% ADD TICKS AFTER FILL

% problem{position(0-18)}{depth}{label}: adds a thicker dark [base colour] tick at position

% emptyproblem: as above, no label or layer

% reference{start}{end}{label}: adds overbrace with caption over top; use for citing where the works are from

% probreference{pos}{extra height}{problem}{reference}: adds line with two line caption on top, use for citing examples of tight classes. extra height might be required for spacing

% multiproblem{start}{end}{depth}{xshift}{label line 1}{label line 2}: adds [base colour] underbrace for infinitely dense classes

% drawtick, tick, emptytick, addbaseticks <- DEPRECATED TO REMOVE

%%% CODE

% scaled pictures
\makeatletter
\newsavebox{\measure@tikzpicture}
\NewEnviron{scaletikzpicturetowidth}[1]{%
  \def\tikz@width{#1}%
  \def\tikzscale{1}\begin{lrbox}{\measure@tikzpicture}%
  \BODY
  \end{lrbox}%
  \pgfmathparse{#1/\wd\measure@tikzpicture}%
  \edef\tikzscale{\pgfmathresult}%
  \BODY
}
\makeatother

%\newenvironment{compline}[1][0.99]{%
\newenvironment{compline}{%
    \begin{center}%
    \begin{scaletikzpicturetowidth}{0.99\textwidth}%
    \begin{tikzpicture}[scale=\tikzscale]%
    \drawbaseunknown%
    }{%
    \end{tikzpicture}%
    \end{scaletikzpicturetowidth}%
    \end{center}%
    }
    
\newcommand{\emptyproblem}[1]{\draw[ultra thick, darkbase] (#1,0.5) -- (#1,0);}

\newcommand{\problem}[3]{\draw[ultra thick, darkbase] (#1,0.5) -- (#1,0) -- ++($(0,-2ex)-(0,#2)$);%
    \node (a) at ($(#1,-#2)-(0,3.5ex)$) {\contour{white}{#3}};%
    }
% ^ tried to make these both into a single 4 argument command, and it broke for some reason. So they're different for now

\newcommand{\multiproblem}[6]{%
    \draw[decorate,decoration={brace,raise=0.5ex,amplitude=2ex}, thick, red] (#2,0) --  (#1,0);%
    \coordinate (CENTER) at ($(#1,-0.5)!0.5!(#2,-0.5)+(0,0.5ex)$);%
    \draw[thick,red] (CENTER) -- ++(0,-0.1) -- ++(0,-#3);%
    \coordinate (LABEL) at ($(#1,0)!0.5!(#2,0)+(0,-0.8)+(#4,-#3)$);%
    \node (a) at (LABEL) {\contour{white}{#5}};%
    \node (b) at ($(LABEL)-(0,3ex)$) {\contour{white}{#6}};%
    }

\newcommand{\reference}[3]{%
    \draw[decorate,decoration={brace,raise=0.5ex,amplitude=2ex}, thick] (#1,0.5) --  (#2,0.5);%
    \coordinate (CENTER) at ($(#1,1)!0.5!(#2,1)-(0,0.15ex)$);%
    \coordinate (LABEL) at ($(#1,1)!0.5!(#2,1)+(0,0.2)$);%
    \draw[thick] (CENTER) -- (LABEL);%
    \node (a) at (LABEL) {\contour{white}{#3}};%
    }

\newcommand{\probreference}[4]{%
    \coordinate (LABEL) at ($(#1,1.5)+(0,#2)$);%
    \draw[thick] (#1,0.5) -- (LABEL);%
    \node (a) at ($(#1,#2)+(0,1.5)+(0,4ex)$) {\contour{white}{#3}};%
    \node (a) at ($(#1,#2)+(0,1.5)+(0,1ex)$) {\contour{white}{#4}};%
    }

\newcommand{\baseticks}[1]{%
    \tick{0}{#1}{$O(1)$}%
    \tick{6}{#1}{$\Theta(\log^*n)$}%
    \tick{12}{#1}{$\Theta(\log n)$}%
    \tick{18}{#1}{$\Theta(n)$}%
}

\newcommand{\dense}[2]{%
    \draw[fill=red, draw=none] ($(#1,0)+(0.075mm,0.075mm)$) rectangle ($(#2,0.5)-(0.075mm,0.075mm)$);%
}

\newcommand{\gap}[2]{%
    \draw[fill=white, draw=none] ($(#1,0)+(0.075mm,0.075mm)$) rectangle ($(#2,0.5)-(0.075mm,0.075mm)$);%
}

\newcommand{\unknown}[2]{%
    \pgfsetfillpattern{crosshatch dots}{base}%
    \filldraw[draw=none] (#1,0) rectangle (#2,0.5);%
}

\newcommand{\drawbase}{%
    \draw (0,0) rectangle (18,0.5);%
}

\newcommand{\drawbaseunknown}{%
    \pgfsetfillpattern{crosshatch dots}{base}%
    \filldraw (0,0) rectangle (18,0.5);%
}

\begin{document}

\section{Definitions}
% General note: I wrote this all out as an exercise in getting my definitions straight, it
% does not need to be THIS detailed.

\subsection{Notation}
We will use $\delta(v)$ to refer to the set of edges adjacent to $v$. We identify
half-edges in a graph $G=(V,E)$ with a pair $(e,v)$ where $e\in\delta(v)$, referring to
the half of $e$ adjacent to $v$.

We use $\MSet{\star}$ to refer to multisets, and the following notation for multisubsets
of a set $A$:
\begin{itemize}
    \item $A^{\MSet{k}}$ refers to the set of all multisubsets of $A$ of size $k\in\N$,
    \item $A^{\MSet{\N}}$ refers to the set of all finite multisubsets of $A$,
    \item $A^{\MSet{\star}}$ refers to the set of all multisubsets of $A$.
\end{itemize}
\subsection{Common and uncommon assumptions}
%\airquotes{Unless otherwise \emph{explicitly} stated...}
\textbf{$T$-hop neighbourhoods:} we define a \emph{$T$-hop centered neighbourhood} as a
\emph{centered graph of radius $T$} using the definitions from page 4 of
\cite{naor_what_1993}. Note that, in the case of subgraphs, the $T$-hop neighbourhood of
$v$ may be different from the subgraph induced by all nodes that have distance at most $T$
from $v$.

\textbf{Local model:} we work in the \emph{deterministic} LOCAL model with input labels.
No assumptions are made about the number of input or output labels. We assume the nodes
are aware of the exact number $n$ of nodes in the input graph, but not of any information
about the maximum degree $\Delta$. Because of this information, we equivalently describe a
LOCAL algorithm as a $T(n)$-round communication algorithm with unbounded message size, or
as a \emph{possibly uncomputable} function from $T(n)$-hop centered neighbourhoods to
output labels. Additionally, we assume there is a finite integer $c$ such that every node
is assigned a unique ID in the set $\Set{1,\ldots,n^c}$, but the nodes are not aware of
the value of $c$.

\textbf{Solvability:} we say that a problem is \emph{weakly unsolvable} if there are
finitely many unsolvable instances for it. We will treat weakly unsolvable problems as
solvable problems by adding one output label $U$ and requiring the problem to output $U$
on all nodes if the instance is unsolvable; this requires only constant time, so it does
not affect asymptotic complexity. We call the maximum diameter of an unsolvable instance
the \emph{solvability horizon} of the weakly unsolvable problem. We say that a problem is
\emph{strongly unsolvable} if there are infinitely many unsolvable instances for it.
%note: some further nuance could be given for "infinitely many solvable and unsolvable
%instances" eg, 2 colouring on cycles, but it is not relevant to the work I want to put in
%here. similarly, "verifiably unsolvable" for infinitely many unsolvable instances that
%can be checked in constant time by every node, but I still have no examples of this (I
%think there are none in LCLs but could be wrong).

\textbf{LCL problems:} we define LCL problems as tuples
$\Pi=(\Sigma\uin,\Sigma\uout,r,\mathcal{C})$ where:
\begin{itemize}
    \item $\Sigma\uin$ is a set of input labels,
    \item $\Sigma\uout$ is a set of output labels,
    \item $r$ is a finite integer, called the \emph{radius} of $\Pi$, and
    \item $\mathcal{C}$ is a \emph{finite} set of $r$-hop centered neighbourhoods, where
    each node is labelled by a pair in $\Sigma\uin\times\Sigma\uout$.
\end{itemize}
Though we haven't explicitly stated this in the definition, WLOG we can and will also
assume that $\Sigma\uin$ is finite (else the problem would be strongly unsolvable) and
that $\Sigma\uout$ is finite (any labels that don't appear in $\mathcal{C}$ will not
appear in any valid solution and can be ignored).
%instance and solution as usual.
\subsection{New-ish definitions}
%these are definitions that have appeared before (cited) but are not "commonly used" as
%far as I could tell or are natural variants of common definitions
\textbf{Hanging trees:} \emph{connected} tree graphs containing a finite number of
incomplete half-edges, called \emph{hooks}. Specifically, we will call hanging trees with
one hook \emph{ornaments} and hanging trees with two hooks \emph{tinsels}.
% I'm using christmasy terms for things because they make a nice and intuitive visual, but
% also because "hanging trees" could have... a different connotation.

\textbf{Class:} given an ornament $T$ and a node-edge checkable problem $\Pi$, we define
the \emph{class} of $T$ to be the set $\mathfrak{Cl}(T)\subseteq \Sigma\uout$ containing
exactly the labellings of the hook that can be completed to a valid labelling for $T$
according to $\Pi$. Up to the changed role of nodes and edges, this is the \emph{class}
definition from Section 3 of \cite{chang_time_2017}.
%I have been calling them types, but chang-pettie calls them classes and uses types for
%paths... I'm trying to stick to classes but there might be mistakes TODO classes for
%ornaments/general hanging trees. Mostly concerned about keeping track of which label goes
%where in a nice way. Leaving it for if I try to generalise the proof for non-trees.

\subsection{Actually new definitions}
%There's some discussion I want to make about writing the def below in "round eliminator
% formalism", but for now it's closer to node-edge checkable the "Extended" is the more
% formally complicated but intuitive definition; the normal FSL definition is formally
% simpler, just as powerful, but not as easy to understand
\textbf{Finitely Represented Configuration:} given an alphabet of symbols $\Sigma$, we
call a \emph{finitely represented configuration} of $\Sigma$ a pair of a \emph{finite}
multisubset of $\Sigma$ (the \emph{requirement}) and a \emph{finite} subset of $\Sigma$
(the \emph{filler}). We denote the set of finitely represented configurations in $\Sigma$
as 
$$\Sigma^{FSL}:=\Sigma^{\MSet{\N}}\times \left[\Sigma\right]^{<\omega}$$

\textbf{Extended Finite State Labelling problems:} we define Extended FSL problems as
tuples $\Pi=(\Sigma\uin,\Sigma\uout,f,\Nc,\Ec)$ where:
\begin{itemize}
    \item $\Sigma\uin$ is a \emph{finite} set of input labels,
    \item $\Sigma\uout$ is a \emph{finite} set of output labels,
    \item $f:\N\to2^{\Sigma\uin}$ is a function describing which input labels are allowed
    to appear for a node of degree $d\in\N$,
    \item $\Nc:\Sigma\uin\to 2^{(\Sigma\uout)^{FSL}}$ is a function assigning to each
    input label a set of \emph{finitely represented configurations} of $\Sigma\uout$, and 
    \item $\Ec$ is a set of pairs of elements of $\Sigma\uout$.
\end{itemize}
An \emph{instance} for $\Pi$ is a graph $G=(V,E)$ together with a labelling function
$l:V\to\Sigma\uin$ such that $l(v)\in f(\deg(v))$, that is, the labelling is coherent with
the \airquotes{allowed} labels for each degree.\\
A \emph{solution} for this instance is a labelling $s$ of the half-edges of $G$ such that:
\begin{itemize}
    \item $\forall v\in V$, let $S(v):=\MSet{s(e,v)\dv e\in\delta(v)}$; then there exists
    $(R,F)\in\Nc(l(v))$ such that $R\subseteq S(v)$ and $(S(v)\smallsetminus R)\subseteq
    F^{\MSet{\star}}$, that is to say, every symbol in the requirement appears exactly
    once in $S(v)$, and every other symbol of $S(v)$ is a filler symbol, and 
    \item $\forall e=\Set{u,v}\in E$ we have $\Set{s(e,u),s(e,v)}\in\Ec$.
\end{itemize}

\textbf{Finite State Labelling problems:} we define FSL problems as tuples
$\Pi=(\Sigma\uout,\Nc,\Ec)$ where:
\begin{itemize}
    \item $\Sigma\uout$ is a \emph{finite} set of output labels,
    \item $\Nc\subseteq 2^{(\Sigma\uout)^{FSL}}$ is a \emph{finite} set of finitely
    represented configurations of $\Sigma\uout$,
    \item $\Ec$ is a set of pairs of elements of $\Sigma\uout$.
\end{itemize}
An \emph{instance} for $\Pi$ is any graph; a solution for this instance is a labelling $s$
of the half-edges of $G$ such that:
\begin{itemize}
    \item $\forall v\in V$, let $S(v):=\MSet{s(e,v)\dv e\in\delta(v)}$; then there exists
    $(R,F)\in\Nc$ such that $R\subseteq S(v)$ and $(S(v)\smallsetminus R)\subseteq
    F^{\MSet{\star}}$, that is to say, every symbol in the requirement appears exactly
    once in $S(v)$, and every other symbol of $S(v)$ is a filler symbol, and 
    \item $\forall e=\Set{u,v}\in E$ we have $\MSet{s(e,u),s(e,v)}\in\Ec$.
\end{itemize}

\textbf{Minimum degree of a label:} we define the \emph{minimum degree} of an input label
$\chi\in\Sigma\uin$ as $\deg_{\min}(\chi):=\min\Set{d\in\N\dv \chi\in f(d)}$ (the minimum
degree for which that input label is allowed). If this set is empty, i.e. $\chi$ is not
allowed for any degree, we assign $\deg_{\min}(\chi):=0$ and observe that the problem would
not change complexity if $\chi$ was removed from the problem description. Generally, we
can assume that $\forall (R,F)\in \Nc(\chi)$ we have $\card{R}=\deg_{\min}(\chi)$, since a
configuration with a longer requirement would be unattainable by a finite number of degrees (which can
be encoded by putting them in separate input classes) and any configuration with a shorter
requirement implicitly requires using a number of filler symbols, which can be encoded by
adding all combinations of filler symbols to the requirements multiset.

\textbf{Width:} we define the \emph{width} of a EFSL problem $\Pi$ as the maximum length of one
of its requirements: accounting for hidden requirements encoded in the degree, we get
$$\max\Set{\max_{\chi\in\Sigma\uin}\deg_{\min}(\chi),\max\Set{\card{R}\dv\exists\chi\in\Sigma\uin,\exists
F\subset \Sigma\uout:(R,F)\in\Nc(\chi)}} $$
For FSL problems we define it as
$\max\Set{\card{R}\dv\exists F(\subset \Sigma\uout:R,F)\in\Nc}$.

\textbf{Height:} we define the \emph{height} of an EFSL problem as
$h(\Pi):=\max\Set{\card{\Nc(\chi)}\dv\chi\in\Sigma\uin}$ (the maximum number of configurations any
node can be in). For FSL problems we define it as $\card{\Nc}$.

\textbf{Restricted (E)FSL:} we define the \emph{restriction} of an (E)FSL the
\emph{node-edge checkable} problem with input labels obtained by replacing every finitely
represented configuration $(R,F)$ with the configurations $\Set{R\cup\MSet{x}\dv x\in F}$.


\section{List of results}
\begin{enumerate}
    \item Extended FSL = FSL (up to changing the number of labels) %we can encode
    input/degree info in the structure and output labels
    \item Node-edge checkable problems $\subseteq$ FSL
    \item Weak and strong solvability are decidable on FSL
\end{enumerate}
\section{List of possible results}
\begin{enumerate}
    \item If a FSL problem is known to be $\Omega(\log n)$ on trees, its complexity on
    trees is decidable and an efficient algorithm can be found. (This holds because of
    Gustav's proof which should be stronger, it's in here because I still haven't written
    out all the details of the FSL based one)
    \item (Belief I haven't been able to fully prove) For each FSL problem $\Pi$ there is
    a value $d$ that only depends on the problem description (specifically, number of
    labels, height and width) such that there is a worst-case family of graphs for $\Pi$
    of maximum degree $d$. My theory is either width$+1$ or width$\cdot$height (also
    possible: width+height, using one of each filler). Latter is
    required for solvability.
    \item (Investigating) A form of the Round Elimination procedure applies to FSL
    problems.
\end{enumerate}


\section{Proofs} 
\subsection{Equivalence of EFSL and FSL}
In this section, we show that the EFSL and FSL can encode the same problems. 
\begin{thm}
    For each FSL problem $\Pi$ that has a LOCAL algorithm $A$ with complexity $T(n)$,
    there is an EFSL problem $\Pi'$ which has a LOCAL algorithm $A'$ with complexity $O(T(n))$.
\end{thm}
\begin{proof}
    Let $\Pi=(\Sigma\uout,\Nc,\Ec)$. We define
    $\Pi'=(\Sigma\uin',\Sigma\uout',f',\Nc',\Ec')$ as follows:
    \begin{itemize}
        \item $\Sigma\uin'=\Set{\star}$,
        \item $\Sigma\uout'=\Sigma\uout$,
        \item $f'$ is the constant function $f(n)=\Set{\star}$,
        \item $\Nc'(\star)=\Nc$,
        \item $\Ec'=\Ec$.
    \end{itemize}
    It is trivial to observe that this is the same problem, and can be solved with the
    same complexity as $\Pi$.
\end{proof}
\begin{thm}
    For each EFSL problem $\Pi$ that has a LOCAL algorithm $A$ with complexity $T(n)$,
    there is an FSL problem $\Pi'$ which has a LOCAL algorithm $A'$ with complexity
    $O(T(n))$.
\end{thm}
This is the more complex of the two proofs, so we provide an overview before getting into
the technical details: we encode the $k$-th input label as a line of degree $3$ nodes of
length $k$, and add leaves as needed to make sure every node of the original graph is of
degree $\geq 4$. This at most multiplies the number of nodes by $3\card{\Sigma\uin}$, and
every node can in finite time compute its own input label by looking at its
$(\card{\Sigma\uin}+1)$-hop neighbourhood. To make sure 
\begin{proof}
    Let $\Pi=(\Sigma\uin,\Sigma\uout,f,\Nc,\Ec)$. Up to a bijection we can identify the
    set $\Sigma\uin$ with the set $\Set{1,\ldots,k}$.
    We define $\Pi'=(\Sigma\uout',\Nc',\Ec')$
    as:
    \begin{itemize}
        \item $\Sigma\uout'=\Sigma\uout\cup \Sigma_I$, where $\Sigma_I:=\Set{(I,n)\dv n\in\Sigma\uin}\cup\Set{(I,L),(I,Err),Err}$,
        \item $\Ec'=\Ec\cup\Set{\MSet{X,X}\dv X\in \Sigma_I}$ (every edge containing the
        new input labels is symmetric).
    \end{itemize}
    $\Nc'$ has a more complex formula, so we provide an algorithm to derive it from $f$
    and $\Nc$ instead:
    % I will write this with the pseudocode package at some point. This is just to check
    % that everything works.
    \begin{enumerate}
        \item %HUH. there is something EFSL can encode that FSL cannot: the fact that some
        % labels are excluded by the function f. I cannot encode, for example,
        % let all odd degree nodes do this and all even degree nodes do that, even if
        % neither depends on degree (eg: weak 2 colouring for odd degrees only). I thought
        % I could do it by giving it an input label, but I would have to either have
        % infinite input labels (encoding all possible degrees) or have instances that
        % might be correct other than the wrong input labels, which could be harder (weak
        % 2col is one such example)
    \end{enumerate}
\end{proof}

\section{What do we know about the complexity landscape?}
\subsection{Trees}
\input{Pictures/Trees}
\subsection{General graphs}

\pagebreak
\printbibliography[ heading=bibintoc, title={Bibliography}]
\end{document}


%% OLD STUFF I REMOVED generally ignore anything below here

%star/filling boxes problem leaf/branch label for a constraint hanging tree types

\pagebreak
\section{The very rough idea}
I have the following algorithms:
\begin{itemize}
    \item \textsf{Hanger} finds all possible hanging tree types, as well as keeping track
    of which ones are only finite and which are not.
    \item \textsf{Path} decides whether a node-edge checkable problem on paths with input
    labels is: 
\begin{itemize}
    \item weakly solvable with complexity $O(\log n)$
    \item weakly solvable with complexity $\Theta(n)$
    \item strongly unsolvable
\end{itemize}    
\end{itemize} 
I want to use them to decide the complexity of a $\Omega(\log n)$ node-edge checkable
problem on bounded degree trees. Plan: use rake-and-compress, solving path problems as you
go, keeping the global ones out. if you can do it till the end, it's O(log n) if you get a
forced global immediately it's O(n) if you do it after k steps it's O(n 1/k).